\section{Ivory Tools}
\label{sec:tools}

Ivory contains built-in tools to support high-assurance software
development. These tools include a correctness condition generator, a SMT-based
symbolic simulator, a theorem-prover translator, and a QuickCheck engine. We
describe each of them briefly below.

None of these tools specifically depend on Ivory being an EDSL, and their
implementations are straight-forward.

\subsection{Correctness Conditions}
Some correctness properties (e.g., arithmetic underflow and overflow properties)
cannot be embedded in the Haskell type system. For these kind of properties, the
Ivory AST is instrumented with appropriate assertions. For example, for the
function

\begin{code}
main :: Def ('[Sint32, Sint32] :-> Sint32)
main = proc "main" $ \ x y -> body $
  ret (x + y)
\end{code}
\noindent
The following function is generated, which contains an overflow check before the
expression is evaluated:
\begin{code}
int32_t main(int32_t var0, int32_t var1)
{
    bool i_ovf0 = add_ovf_i32(var0, var1);
    COMPILER_ASSERTS(i_ovf0);
    return (int32_t) (var0 + var1);
}
\end{code}
\noindent
The overflow check is performed by the function
\cd{add\_ovf\_i32}. \cd{COMPILER\_ASSERTS} is a C macro; the appropriate response when
property is violated is platform-dependent (e.g., logging, do-nothing, run a
recovery procedure, etc.).

Ivory inserts correctness condition checks for the following, as requested by
the user:
\begin{itemize}
\item no arithmetic underflow and overflow,
\item no division-by-zero,
\item no bit-shifts are greater than or equal to the value's width,
  (bit-shifts on signed integers is prevented statically by the type system)
\item no floating-point operations result in \cd{inf} or \cd{NaN} values.
\end{itemize}

The implementation is mostly straightforward; the two aspects that require some
care of are sharing and control-flow expressions. As an EDSL, Ivory encourages
the use of macros which can result in large expressions. In this case, each
subexpression must be checked. Standard common subexpression elimination can
dramatically reduce the number of instrumented assertions. Second, Ivory contains
the short-cutting expressions of conjunctions, disjunctions, and conditionals,
analogs of C's \cd{\&\&}, \cd{||}, and \cd{\_ ? \_:\_}, respectively. For these
expressions, the generated assertions must contain as a precondition that
short-cutting has not occurred so as not to be overly pessimistic. For example,
for the expression
\noindent
\begin{code}
x != 0 ? 3/x : 0
\end{code}
\noindent
the assertion
\begin{code}
(x == 0) || (x != 0)
\end{code}
\noindent
is generated (and in this case, subsequently discharged by the constant folder).

Generated correctness conditions as well as general user assertions can be
discharged via testing or using model-checking or theorem-proving. We describe
Ivory's tooling for these approaches below.

\paragraph{Symbolic Simulation}
Ivory contains a symbolic simulator built over CVC4~\cite{} for verifying
programs. Ivory programs are typically amenable to formal analysis since they
guarantee the absence of memory-safety errors, there is no pointer arithmetic,
there is no heap, and they are not concurrent. Given a set of preconditions, the
simulator attempts to verify any inline assertions and postconditions. The
simulator ensures that the collection of preconditions are satisfiable.

The symbolic simulator abstracts various domains. Floating point types are
abstracted as reals. However, fixed-width values are modeled precisely as are
arrays. We have not yet incorporated a theory of bit-vectors yet,
however. \eric{The last two sentences may be a bit confusing, since you'd usually expect a precise encoding of fixed-width integers to use bit-vectors. Of course we get away with using integers since we strictly disallow overflow. Perhaps we should clarify here?} Non-linear operators are abstracted with linear contracts. Loops are
unrolled (all loops in Ivory have a constant upper bound).

During analysis, inter-procedural calls are inlined or abstracted based a on
user-supplied option. If they are abstracted, then the callee's precondition is
added as a verification condition at the call site. The callee's postcondition
is added to the set of invariants following the call. For imported procedures
hand-written in C, Ivory allows them to be augmented with pre- and
postconditions to abstract them.

The simulator is simple compared to other software
model-checkers\cite{cbmc,cpachecker,ultiautomizer}, providing only bounded
model-checking, no support for concurrency, and no counter-example guided
refinement, matching the simplicity of the language.

We have used the symbolic simulator to analyze various Ivory programs, having
found an off-by-one bug in a ring buffer and verifying the correctness of a
safety state-machine in SMACCMPilot~\cite{}.

\paragraph{Theorem Proving}

Eakman~et~al. has implemented a theorem-prover back-end for
Ivory~\cite{ivory-acl2} that targets ACL2~\cite{acl2}. The theorem-prover
performs inter-procedural analysis, abstracting procedure calls by their
contracts, like the symbolic simulator. However, the theorem prover has the
ability to scale more substantially, at the price of doing interactive
proofs. Eakman~et~al. provide examples of the use of the theorem-proving
back-end.

\paragraph{Property-Based Testing}

Finally, we have implemented a QuickCheck~\cite{} like property-based testing
framework for Ivory. The framework tests procedures by randomly generating
values both for their formal arguments as well as for global values referenced
by the procedure. Ivory is primarily a compiled language, so tests are compiled
to C to be executed. Only values that satisfy procedure preconditions should be
generated, so the framework contains an interpreter for evaluating values before
code generation.


