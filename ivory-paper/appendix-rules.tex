
\section{Ivory Semantics Rules}
\label{sec:all-semantics}

In this section we give the semantics of Ivory.  This is a simplification of the Isabelle/HOL development, available in the
ivory language distribution.

\subsection{Syntax}

\[
\begin{array}{crl}
\syntaxtitle{pure expressions}\\
e & ::= & 0 \sep{} 1 \sep{} \ldots{} \sep{} \texttt{true} \sep{} \texttt{false} \sep{} () \sep{} x \sep{} 
          e_1 \mathbin{\mathit{op}} e_2 \\
\syntaxtitle{impure expressions}\\
i & ::= & \texttt{pure}(e) \sep{} \texttt{alloc}(e) \sep{}
          \texttt{read}(e) \sep{} \texttt{write}(e_1, e_2)\\
\syntaxtitle{statements}\\
s & ::= & \texttt{skip} \sep{} \texttt{return}(e) \sep{} s_1; s_2 \\ 
  & |   & \texttt{if}(e)\texttt{\;then\;} s_1 \texttt{\;else\;} s_2 \\
  & |   & \texttt{for}(x = e_1; e_2; e_3) \{ s \} \\
  & |   & \texttt{let\;} x = i \texttt{\;in\;} s \\
  & |   & \texttt{let\;} x = f(e_1, \ldots{}, e_n) \texttt{\;in\;} s \\
\syntaxtitle{values}\\
v & ::= & 0 \sep{} 1 \sep{} \ldots{} \sep{} \texttt{true} \sep{} \texttt{false} \sep{} () \\
w & ::= & \texttt{stored}(v) \sep{} \texttt{ref}(r, n) \\
\syntaxtitle{Stores}\\
E & \in{} & x \to w \\
\syntaxtitle{regions and heaps}\\
R & \in{} & \mathbb{N} \to w\\
H & ::=   & H, R \sep{} \emptyset{}\\
\syntaxtitle{stacks}\\
F & ::= & \texttt{rframe}(x, E, s) \sep{} \texttt{sframe}(E, s) \\
S & ::= & F, S \sep{} \emptyset{} \\
\syntaxtitle{configurations}\\
C & \in & H \times S \times E \times s \sep{} \texttt{finished}(v) \\
\syntaxtitle{types}\\
\rho   & \in & \textit{region variables}\\
\alpha & ::= & \texttt{nat} \sep{} \texttt{bool} \sep{} \texttt{unit} \\
\tau   & ::= & \texttt{storedt}(\alpha) \sep{} \texttt{reft}(\rho, \alpha) \\
\syntaxtitle{procedure definitions}\\
P & ::= & \texttt{proc\;} f(\tau_1\;x_1, \ldots{}, \tau_n\;x_n) : \tau\; \{ s \} \\
\texttt{Procs} & \in & \texttt{pow}(P)
\end{array}
\]

\subsection{Abstract machine}

The semantics of Ivory is encoded as an abstract machine.  

\paragraph{Pure expressions} Pure expresses are given a denotational semantics, as below.

\[
\begin{array}{rcl}
\denoteexp{i}{E}  &=&i \\
\denoteexp{b}{E}  &=&b \\
\denoteexp{()}{E} &=&() \\
\denoteexp{x}{E}  &=&E\;x \\
\denoteexp{e_1 \mathbin{\mathit{op}} e_2}{E} &=& \denoteexp{e_1}{E} \mathbin{\mathit{op}} \denoteexp{e_2}{E} \\
\end{array}
\]

\paragraph{Impure expressions}
The semantics of impure expressions operates over heaps and impure expressions, in the context of a
store.  The judgement
\[
\hsteps{E}{H}{i}{H}{v}
\]
holds when impure expression $i$ in heap $H$ evaluates to heap $H'$ and result $v$.

\begin{mathpar}
\inferrule{\denoteexp{e}{E} = w}{\hsteps{E}{H}{\texttt{pure}(e)}{H}{w}}
\and
\inferrule{\denoteexp{e}{E} = \texttt{stored}(v)}{\hsteps{E}{H, R}{\texttt{alloc}(e)}{H, R[p \mapsto{} v]}{\texttt{ref}(|H|, p)}}(p \notin{} \mathrm{dom}(R))
\and
\inferrule{\denoteexp{e}{E} = \texttt{ref}(r, n)  \\ (r, n) \in \texttt{dom}(H)}{ \hsteps{E}{H}{ \texttt{read}(e) }{H}{\texttt{stored}(H(r, n))} }
\and
\inferrule{\denoteexp{e_1}{E} = \texttt{ref}(r, n) \\ (r, n) \in \texttt{dom}(H) \\ \denoteexp{e_2}{E} = \texttt{stored}(v)}{ \hsteps{E}{H}{ \texttt{write}(e_1, e_2) }{H[(r, n) \mapsto v]}{\texttt{stored}(())} }
\end{mathpar}

\paragraph{Statements}
The judgement
\[
\stepsX{C}{C'}
\]
states that configuration $C$ transitions to configuration $C'$.  A
configuration consists of a heap, a stack, a store, and the current
statement. The stack contains continuations for both function calls
and statement sequences while the store maps variables to values. The
semantics of sequencing is slightly non-standard as variables are
let-bound rather than assigned, and so statement sequencing preserves
the store across execution of the first statement.

%% Statements
\begin{mathpar}
\inferrule{ {} }{\steps{(H; \texttt{sframe}(E, s), S; \_)}{\texttt{skip}}{(H; S; E)}{s}}
\and
\inferrule{ {} }{\steps{(H; \texttt{sframe}(\_, \_), S; E)}{\texttt{return}(e)}{(H; S; E)}{\texttt{return}(e)}}
\and
\inferrule{ \denoteexp{e}{E_0} = w }{\steps{(H, \_; \texttt{rframe}(x, E, s), S; E_0)}{\texttt{return}(e)}{(H; S; E[x \mapsto{} w])}{s}}
\and
\inferrule{ \denoteexp{e}{E} = w }{ \stepsX{(\_; \emptyset; E); \texttt{return}(e)}{\texttt{finished}(w)} }
\and
\inferrule{ {} }{\steps{(H; S; E)}{ (s_1; s_2) }{(H; \texttt{sframe}(E, s_2), S; E)}{s_1}}
\and
\inferrule{ \denoteexp{e}{E} = \texttt{stored}(\texttt{true}) }{\steps{(H; S; E)}{ \texttt{if}(e)\texttt{\;then\;} s_1 \texttt{\;else\;} s_2 }{(H; S; E)}{s_1}}
\and
\inferrule{ \denoteexp{e}{E} = \texttt{stored}(\texttt{false}) }{\steps{(H; S; E)}{ \texttt{if}(e)\texttt{\;then\;} s_1 \texttt{\;else\;} s_2 }{(H; S; E)}{s_2}}
\and
\inferrule{ \denoteexp{e_1}{E} = w }{\steps{(H; S; E)}{ \texttt{for}(x = e_1; e_2; e_3) \{ s \} }{(H, S, E[x \mapsto{} w])}%
{\texttt{if}(e_2)\texttt{\;then\;} (s; \texttt{for}(x = e_3; e_2; e_3) \{ s \}) \texttt{\;else\;} \texttt{skip} }}
\and
\inferrule{ \hsteps{E}{H}{i}{H'}{w} }{\steps{(H; S; E)}{\texttt{let\;} x = i \texttt{\;in\;} s}{(H'; S; E[x \mapsto{} w])}{s}}
\and
\inferrule{ \denoteexp{e_i}{E} = w_i \\ \texttt{proc\;} f(\tau_1\;x_1, \ldots, \tau_n\;x_n) : \tau\;\{ \mathit{body} \} \in \texttt{Procs} }{\steps{(H; S; E)}{\texttt{let\;} x = f(e_1, \ldots{}, e_n) \texttt{\;in\;} s}{(H, \emptyset; \texttt{rframe}(x, E, s), S; [x_1 \mapsto{} w_1. \ldots{}, x_n \mapsto{} w_n])}{\mathit{body}}}

\end{mathpar}

\subsection{Typing rules}

\paragraph{Pure expressions} Pure expression typing is standard, with $\mathit{op}_\tau$ an
operation that returns type $\tau$, where $\tau$ is \texttt{nat} for
arithmetic operations and \texttt{bool} for comparison operations.

\begin{mathpar}
\inferrule{  }{ \wfexp{\Gamma}{i}{\texttt{nat}} }\and
\inferrule{  }{ \wfexp{\Gamma}{b}{\texttt{bool}} }\and
\inferrule{  }{ \wfexp{\Gamma}{()}{\texttt{unit}} }\and
\inferrule{ \Gamma(x) = \tau }{ \wfexp{\Gamma}{x}{\tau} }\and
\inferrule{ \wfexp{\Gamma}{e_1}{\texttt{nat}} \\ \wfexp{\Gamma}{e_2}{\texttt{nat}} }%
{\wfexp{\Gamma}{e_1 \mathbin{\mathit{op}_\tau} e_2}{\tau}}
\end{mathpar}

\paragraph{Impure expressions}
The typing judgement for impure expressions, as with statements, is
parameterized by the current region, $\rho$.  

\begin{mathpar}
% wfimp
f\inferrule{ \wfexp{\Gamma}{e}{\tau} }{ \wfimp{\Gamma}{\rho}{\texttt{pure}(e)}{\tau} }
\and
\inferrule{ \wfexp{\Gamma}{e}{\alpha} }{ \wfimp{\Gamma}{\rho}{\texttt{alloc}(e)}{\texttt{reft}(\rho, \alpha)} }
\and
\inferrule{ \wfexp{\Gamma}{e}{\texttt{reft}(\rho, \alpha)} }{ \wfimp{\Gamma}{\rho}{\texttt{read}(e)}{\texttt{storedt}(\alpha)} }
\and
\inferrule{ \wfexp{\Gamma}{e_1}{\texttt{reft}(\rho, \alpha)} \\ \wfexp{\Gamma}{e_2}{\alpha} }{ \wfimp{\Gamma}{\rho}{\texttt{write}(e_1, e_2)}{\texttt{storedt}(\texttt{unit})} }
\end{mathpar}

\paragraph{Statements}
The typing judgement 
\[
\wfstmt{\Gamma}{\Psi}{\rho}{s}{\tau}
\]
holds when the statement $s$ is well-formed under the store
environment $\Gamma$, current region $\rho$, with any return
statements returning values of type $\tau$.

\begin{mathpar}
%% wfstmt
\inferrule{ }{ \wfstmt{\Gamma}{\Psi}{\rho}{\texttt{skip}}{\tau} }
\and
\inferrule{ \wfexp{\Gamma}{e}{\tau}  }{ \wfstmt{\Gamma}{\Psi}{\rho}{\texttt{return}(e)}{\tau} }
\and
\inferrule{ \wfstmt{\Gamma}{\Psi}{\rho}{s_1}{\tau} \\ \wfstmt{\Gamma}{\Psi}{\rho}{s_2}{\tau} }{ \wfstmt{\Gamma}{\Psi}{\rho}{s_1; s_2}{\tau} }
\and
\inferrule{ \wfexp{\Gamma}{e}{\texttt{bool}} \\ \wfstmt{\Gamma}{\Psi}{\rho}{s_1}{\tau} \\ \wfstmt{\Gamma}{\Psi}{\rho}{s_2}{\tau} }%
{ \wfstmt{\Gamma}{\Psi}{\rho}{\texttt{if}(e)\texttt{\;then\;} s_1 \texttt{\;else\;} s_2 }{\tau} }
\and
\inferrule{ \wfexp{\Gamma}{e_1}{\sigma} \\ \wfexp{\Gamma[x \mapsto \sigma]}{e_2}{\texttt{bool}} \\ \wfexp{\Gamma[x \mapsto \sigma]}{e_3}{\sigma}
\\ \wfstmt{\Gamma[x \mapsto \sigma]}{\Psi}{\rho}{s}{\tau}}%
{ \wfstmt{\Gamma}{\Psi}{\rho}{ \texttt{for}(x = e_1; e_2; e_3) \{ s \} }{\tau} }
\and
\inferrule{ \wfimp{\Gamma}{\rho}{i}{\sigma} \\  \wfstmt{\Gamma[x \mapsto \sigma]}{\Psi}{\rho}{s}{\tau} }%
{ \wfstmt{\Gamma}{\Psi}{\rho}{\texttt{let\;} x = i \texttt{\;in\;} s}{\tau} }
\and
\inferrule{ \wfexp{\Gamma}{e_i}{\sigma_i} \\
            \texttt{proc\;} f(\sigma_1\;x_1, \ldots, \sigma_n\;x_n) : \sigma\;\{ \mathit{body} \} \in \texttt{Procs} \\
            \wfstmt{\Gamma[x \mapsto \sigma]}{\Psi}{\rho}{s}{\tau} }%
{ \wfstmt{\Gamma}{\Psi}{\rho}{\texttt{let\;} x = f(e_1, \ldots{}, e_n) \texttt{\;in\;} s}{\tau} }

\end{mathpar}

\paragraph{Programs}
A program is well-formed when all its constituent procedures are
well-formed under a fresh region variable and constrained return type.

\begin{mathpar}
\inferrule{\forall\texttt{proc\;} f(\tau_1\;x_1, \ldots, \tau_n\;x_n) : \tau\;\{ \mathit{body} \} \in \texttt{Procs}\\\\
           \rho \textrm{ fresh} \\     
            \texttt{frees}(\tau) \subseteq \texttt{frees}(\tau_1) \cup \ldots \cup \texttt{frees}(\tau_n) \\
            \wfstmt{[x_1 \mapsto \tau_1, \ldots, x_n \mapsto \tau_n]}{\Psi}{\rho}{s}{\tau} }%
{ \vdash \texttt{Procs} }

\end{mathpar}
