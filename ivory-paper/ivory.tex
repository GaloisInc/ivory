%-----------------------------------------------------------------------------
%
%               Template for sigplanconf LaTeX Class
%
% Name:         sigplanconf-template.tex
%
% Purpose:      A template for sigplanconf.cls, which is a LaTeX 2e class
%               file for SIGPLAN conference proceedings.
%
% Guide:        Refer to "Author's Guide to the ACM SIGPLAN Class,"
%               sigplanconf-guide.pdf
%
% Author:       Paul C. Anagnostopoulos
%               Windfall Software
%               978 371-2316
%               paul@windfall.com
%
% Created:      15 February 2005
%
%-----------------------------------------------------------------------------


\documentclass{sigplanconf}

% The following \documentclass options may be useful:

% preprint      Remove this option only once the paper is in final form.
% 10pt          To set in 10-point type instead of 9-point.
% 11pt          To set in 11-point type instead of 9-point.
% authoryear    To obtain author/year citation style instead of numeric.

\usepackage{ifthen}
\usepackage{color}
\usepackage{framed}
\usepackage{paralist}
\usepackage{mathtools}
\usepackage{listings}
\usepackage{textcomp}
\usepackage{fixltx2e}
\usepackage{url}
\usepackage{flushend}
\usepackage{bold-extra}
%% \usepackage{array} %sjw, for <{$} etc.
\usepackage{hyperref} %sjw, for \autoref
\usepackage{stmaryrd} %sjw, for oxford brackets (\llbracket, \rrbracket)
\usepackage{amssymb}  %sjw, mainly for mathbb

%% For inference rules
\usepackage{mathpartir}

% Language definition for both quasi-quoted Ivory, and Haskell.
\lstdefinelanguage{Ivory}
{morekeywords={struct,data,where,let,in,class,instance,type,family},
 sensitive=true,
 morestring=[b]",
 escapeinside={\%*}{*)},
 commentstyle=\ttfamily,
 morecomment=[l]{--},
 morecomment=[s]{\{-}{-\}}
}

\lstset{language=Ivory}

\newboolean{td}
  \setboolean{td}{true} % modify here
  \ifthenelse{\boolean{td}}
             {\newcommand{\pat}[1]{\textcolor{blue}{PH: #1}}
              \newcommand{\lee}[1]{\textcolor{blue}{LP: #1}}
              \newcommand{\sjw}[1]{\textcolor{blue}{SW: #1}}
              \newcommand{\trevor}[1]{\textcolor{blue}{TE: #1}}
              \newcommand{\james}[1]{\textcolor{blue}{JB: #1}}
             }
             {\newcommand{\pat}[1]{}
              \newcommand{\lee}[1]{}
              \newcommand{\sjw}[1]{}
              \newcommand{\trevor}[1]{}
              \newcommand{\james}[1]{}
             }

\newcommand{\mytilde}{\raise.17ex\hbox{$\scriptstyle\mathtt{\sim}$}}
\lstnewenvironment{code}[1][]
  {\lstset{basicstyle=\scriptsize\ttfamily,#1}}
  {}
%% \lstnewenvironment{smcode}[1][]
%%   {\lstset{basicstyle=\scriptsize\ttfamily,#1}}
%%   {}

%% \newenvironment{cols}{\begin{tabular}{m{3.6cm}|m{3.6cm}} &
%%     \\\hline}{\end{tabular}}

\newcommand{\cd}[1]{\texttt{#1}}

\begin{document}

\special{papersize=8.5in,11in}
\setlength{\pdfpageheight}{\paperheight}
\setlength{\pdfpagewidth}{\paperwidth}

\conferenceinfo{ICFP~'14}{September 1--6, 2014, Gothenburg, Sweden}
\copyrightyear{2014}
\copyrightdata{978-1-4503-2873-9/14/09}
\doi{2628136.2628146}

% Uncomment one of the following two, if you are not going for the 
% traditional copyright transfer agreement.

%\exclusivelicense                % ACM gets exclusive license to publish, 
                                  % you retain copyright

\permissiontopublish             % ACM gets nonexclusive license to publish
                                  % (paid open-access papers, 
                                  % short abstracts)

\titlebanner{Under submission}        % These are ignored unless
\preprintfooter{short description of paper}   % 'preprint' option specified.

\title{Ivory}
\subtitle{catchy subtitle}

\authorinfo{Trevor decides authors and order}
           {Galois, Inc.}
           {\{foobar\}@galois.com}
%% \authorinfo{Name2\and Name3}
%%            {Affiliation2/3}
%%            {Email2/3}

\maketitle

\begin{abstract}
Ivory is a ``safe C'' language, enforcing memory safety and removing most
undefined behaviors, while providing low-level control of memory-manipulation
and timing. Ivory is embedded in a modern variant of Haskell, as implemented by
the GHC compiler. The main contributions of the paper are two-fold. First, we
demonstrate how to embedded type type-system of a safe-c language into the
dependent-type extensions of GHC. Second, Ivory is of interest in its own right,
as a powerful language for writing high-assurance embedded programs. Beyond
invariants enforced by its type-system, Ivory has direct support for
model-checking, theorem-proving, and property-based testing. Ivory's semantics
have been formalized and proved correct.
\end{abstract}

\category{D.3.2}{Language Classifications}{Applicative (functional) languages}

% general terms are not compulsory anymore,
% you may leave them out
%% \terms
%% term1, term2

\keywords
Embedded Domain Specific Languages; Embedded Systems

\section{Introduction}
\label{sec:introduction}

Recent reports of car-hacking via software flaws~\cite{cars11}, and insecure
low-level networking code~\cite{heartbleed} point toward the need for safe low-level programming
languages. Languages like C or C++ are still the gold standard in embedded
system development given the low-level control they provide in terms of memory
usage and timing behavior. Unfortunately, these languages provide
little support for creating high assurance software --- they are
unsafe and unanalyzable.

In this paper we present the language \emph{Ivory}.\footnote{%
Ivory is open-source (BSD3 license) and available at \url{ivory-lang.org}.}
Ivory follows in the
footsteps of other ``safe C'' programming languages, like
Cyclone~\cite{cyclone}, BitC~\cite{bitc}, and Rust~\cite{rust}. By safe C
languages, we mean languages that avoid many of the pitfalls of C, particularly
related to memory safety and undefined behavior, while being suitable for
writing low-level code (e.g., device drivers), and having minimal runtime systems.

Ivory is particularly designed for safety-critical embedded programming. Such a language
should guarantee memory safety, prevent most undefined behaviors, and provide
integrated testing and verification tools. Additional conventions for safe
embedded systems such as those in use at NASA's Jet Propulsion
Laboratory~\cite{mars}, are enforced by Ivory's type system. The major
restrictions enforced by Ivory include restricted allocation so that
zero-overhead garbage collection is feasible, enforcing loops
to have constant upper bounds, avoiding machine-dependent types (e.g., \cd{int}),
and safe dereferencing.

Ivory's implementation, however, is unique compared to previous safe C
languages: Ivory is implemented as an embedded domain-specific language (EDSL)
within Haskell.  In addition to the benefits of rapid language development, this
gives Ivory a powerful templating system---the language Haskell---allowing
low-level programs to be written in a high-level style, despite some of the
languages restrictions.

Ivory's type system is shallowly embedded within Haskell's type
system, taking advantage of the extensions provided
by GHC~\cite{dephaskell}.  Thus, well-typed Ivory programs are
guaranteed to produce memory safe executables, all without writing a
stand-alone type-checker.

In contrast, the Ivory syntax is \emph{deeply} embedded within
Haskell.  This novel combination of shallowly-embedded types and
deeply-embedded syntax permits ease of development without sacrificing
the ability to develop various back-ends and verification tools: in
addition to the generation of embedded C for compilation, the Ivory
language suite includes an integrated SMT-based symbolic simulator and
a theorem-prover back-end.  All these back-ends share the same AST:
Ivory verifies what it compiles.


Ivory is not a toy language: we have used Ivory to write
\emph{SMACCMPilot}~\cite{smaccm}, a full-featured high-assurance
autopilot for a small unpiloted air vehicle.  Furthermore, Boeing has
used Ivory to implement a level-of-interoperability for
Stanag~4586~\cite{stanag}, a unpiloted air vehicle communications
standard. We know of a few additional small projects by other
developers in Ivory, as well.  There are well over 100~KLoC of Ivory in existence.

\sjw{Ivory uses regions for memory management.}

\sjw{move to contributions bit?}

\paragraph{Contributions}
In 2014, Stephanie Weirich gave an ICFP keynote describing dependently-typed programming
using recent type and language extensions of GHC's implementation of Haskell
(henceforth ``GHC'') paper~\cite{weirich-keynote}. Weirich describes how recent extensions to GHC
provide much of the power of dependently-typed programming, such as found in
Agda~\cite{agda}, Idris~\cite{idris}, or Coq~\cite{coq}. However, in GHC, a
surprisingly powerful subset of dependent typing features are
available without sacrificing type-inference and decidable type-checking~\cite{dephaskell}.

Ivory exemplifies the use of GHC dependent types in a large, fully-featured
EDSL. The language demonstrates that type checking for safe systems programming
can be embedded into GHC's type system, verifying properties involving
pointers, arrays, loops and local memory allocation. Indeed, Ivory's type
system goes beyond mere memory safety and tracks control-flow effects at host-language
compile-time.

After providing a brief introduction to the Ivory language in
Section~\ref{sec:ivory-overview}, we describe Ivory's embedding in GHC's type
system in Section~\ref{sec:ivory-embedding}. We highlight the aspects of the
language particularly relevant to memory-safety (e.g., pointers, structures, and
memory allocation). We also highlight shortcomings of the approach, describing
aspects of the language that cannot be checked by the host language's type
system (e.g., Ivory's module system).

Embedding a type system for a safe C language into GHC's type system is tricky
business. To gain confidence that our embedding is correct, we formalized a
model of Ivory in the Isabelle theorem prover~\cite{isabelle}, and used the model to
formally prove progress and preservation properties for Ivory. In the process,
we discovered minor bugs in Ivory's type embedding in GHC as well as
generalizations to Ivory that still preserve safety. We describe the
formalization, proofs, and extensions in Section~\ref{sec:semantics}.

Ivory goes beyond ensuring memory safety, the focus of most other safe C
programming languages, and also provides automated support for preventing errors
that result from other undefined behaviors in C (e.g., division by zero, left
bit-shifts by a negative value, etc.) as well as support for checking
user-provided assertions. Toward this end, Ivory supports writing user-supplied
assertions and pre- and post-conditions on functions, and includes a built-in
symbolic simulator targeting an SMT solver (CVC4~\cite{cvc4}), as well as an
theorem-prover back-end targeting ACL2~\cite{acl2}. For automated testing, a
QuickCheck-like property-based test-case generator is integrated into
Ivory. These tools are described in Section~\ref{sec:tools}. In
Section~\ref{sec:edsl}, we discuss some of the issues and our mitigations with
using a large EDSL for embedded programming projects.

We describe related work in safe C language and EDSLs in
Section~\ref{sec:related-work} and provide concluding remarks in
Section~\ref{sec:conclusion}.





\section{Ivory Overview}
\label{sec:ivory-overview}

\lee{need lead-in here. What does Ivory do? What's the motivation for it's
  design (I like to motivate it with the JPL's Power of 10 rules).}

In this section we give an illustrative overview of Ivory.  An Ivory
program is a Haskell program producing a collection of Ivory modules,
each module containing type and procedure definitions.  Type
definitions are produced using a quasi-quoter\cite{}, while procedure
definitions are built from a set of monadic combinators.

Ivory is a staged language: the Haskell program compiles Ivory modules
to produce an AST which is then passed to one or more back-ends.  Thus,
an executable is produced from an Ivory program by compiling and
running the Haskell code to produce a C program, which is then
compiled with a C compiler.  Alternately, checking of pre- and
post-conditions is performed by simply running the Haskell program in
conjunction with the verification back-end.

\subsection{Ivory Statements}

\lee{I'd suggest we elide types in this section, to the extent possible.}

Ivory statements are terms in the \cd{Ivory} monad.  This monad
provides fresh variables, along with constructors for Ivory
statements. Unlike C, Ivory expressions must be pure, so
side-effecting operations take place at the statement level, in
the context of the monad. This ensures a defined order for effects,
eliminating large classes of unintuitive and undefined behaviors.

Memory in Ivory is manipulated via non-nullable references.
References are read and written using the \cd{deref} and \cd{store}
statements, respectively.  For example, the following Haskell function
takes a reference to a signed 32 bit integer value and constructs
statements which increment the value of the reference.

\begin{code}
incr_ref :: Ref s (Stored Sint32) -> Ivory eff ()
incr_ref r = do
    v <- deref r
    store r (v + 1)
\end{code}

We note that this function is \emph{not} a complete Ivory procedure.
Rather, it can be thought of as a template parameterised by a reference.
Ivory procedures must be explicitly defined and exported through
procedure definitions, such as
\begin{code}
ex_incr_proc :: Def ('[Ref s (Stored Sint32)] :-> Sint32)
ex_incr_proc = proc "ex_incr" $ \r -> body $ do
  incr_ref r
  v <- deref r
  ret r
\end{code}
The Ivory embedding uses standard Haskell techniques\cite{} to reuse
Haskell's binders to name procedure arguments.  These binders do not
appear in the AST, however, being erased by the \cd{proc} smart
constructor.

A reference in Ivory may refer to either a global object, allocated at
compile time, or a \emph{local} object, allocated dynamically.
Dynamic objectes are created in ephemeral \emph{regions} associated
with the scope of the containing procedure; operationally, local
objects are allocated on the stack, so regions are implicitly free'd
on procedure return.  Ivory reference types are indexed by region
variables, the parameter \cd{s} seen in the type signatures above.
Along with type variable scoping, these region annotations on
references ensure that references do not escape the context in which
they were allocated.

The Ivory monad tracks effects such as allocation through a parameter
to the \cd{Ivory} monad.  The allocation effect ties the current
region to the monad, and the allocation function \cd{local} returns a
reference in the current region.  For example, the following
constructs a zero initialized reference to an integer.
\begin{code}
make_zero :: (GetAlloc eff %*\mytilde*) Scope s)
             => Ivory eff (Ref s (Stored Sint32))
make_zero = local (ival 0)
\end{code}

\subsection{Data structures}

Ivory provides C-style arrays and data structures.  Arrays types are
parameterised by their size, the type system ensuring that array accesses
are within bounds.  Data structures are defined using a quasi-quoter
to specify the field names and their types.  Fields in data structures
are accessed via the \cd{\mytilde>} operator which takes a
reference to a struct and a field name, and returns a reference at the
field's type. For example, the following code declares a structure
type named \cd{position}, allocates an initialized instance, and then
shows some basic operations on elements in the structure.

\begin{code}
[ivory|
struct position
  { latitude  :: Stored IFloat
  ; longitude :: Stored IFloat
  ; altitude  :: Stored Sint32
  }
|]

struct_ex :: (GetAlloc eff %*\mytilde*) Scope s) => Ivory eff ()
struct_ex = do
  s <- local (istruct [ latitude .= ival 45.52
                      , longitude .= ival (-122.68)
                      , altitude .= ival 1524 ])
  lat <- deref (s %*\mytilde*)> latitude)
  lon <- deref (s %*\mytilde*)> longitude)
  incr_ref (s %*\mytilde*)> altitude)
\end{code}

\subsection{Control structures}
\label{sec:control}

Ivory supports the usual control structures: the \cd{ifte\_} statement
constructor takes a boolean argument and two statements, one for each
branch of the if-then-else, while the pure ternary operator, \cd{?},
selects from two alternatives at the expression level.

\begin{code}
abs :: Def('[Sint16] :-> Sint16)
abs = proc "abs" $ \v -> body $ do
  ifte_ (v <? 0)
    (ret (-1*v))
    (ret v)

abs2 :: Def('[Sint16] :-> Sint16)
abs2 = proc "abs2" $ \v -> body $ do
  ret $ (v <? 0) ? ((-1*v), v)
\end{code}
%$
Ivory has two iteration constructs: \cd{forever} for non-terminating
loops such as OS tasks, and \cd{arrayMap} which iterates over a range;
as the name suggests, the primary purpose of \cd{arrayMap} is
iteration over elements in an array.  For example, the following
procedure adds \cd{x} to each element of the array \cd{arr}, noting
that \cd{arr ! ix} returns a \emph{reference} to the \cd{ix}-th
element of \cd{arr}.

\begin{code}
mapProc = proc "mapProc"
        $ \arr x -> body
        $ arrayMap
        $ \ix -> do
            v <- deref (arr ! ix)
            store (arr ! ix) (v + x)
\end{code}
%$
Note that we do not need to pass \cd{arr} to \cd{arrayMap} to determine the
correct bounds on the loop; rather, as we explain in \autoref{sec:area}, GHC can
\emph{infer} the bounds from the loop body!

\subsection{Assertions}

Ivory supports pre- and post-conditions, along with assertions.  The
Ivory compiler can emit run-time assertions to enforce these
conditions, or the model checker back-end can be used to statically
verify these properties hold.

\begin{code}
predicates_ex :: Def('[ IFloat ] :-> IFloat)
predicates_ex = proc "predicates_ex" $
    \i -> requires (i >? 0)
        $ ensures (\r -> r >? 0)
        $ body
        $ do (assert (i /=? 0))
             ret (i + 1))
\end{code}


\sjw{removed, what is the main point of this section?}
\jamey{The printf\_proc example illustrates this, but does it pass the sanity
checker? I don't think so...?}
\eric{Good point, the sanity checker will complain about the call to
  printf\_none, but I'd call this a shortcoming of the sanity checker. I guess
  it needs some notion of sum types for importProcs (and maybe externProcs?).}

% \begin{code}

% print_ex_module :: Module
% print_ex_module = package "print_ex" $ do
%   incl printf_none
%   incl printf_sint32
%   incl print_proc

% printf_none :: Def('[IString] :-> Sint32)
% printf_none  = importProc "printf" "stdio.h"

% printf_sint32 :: Def('[IString, Sint32] :-> Sint32)
% printf_sint32  = importProc "printf" "stdio.h"

% print_proc = Def('[]:->())
% print_proc = proc "print_proc" $ body $ do
%   _ <- call printf_none "hello, world!\n"
%   _ <- call printf_sint32 "print an integer: \%d" 42
%   return ()
% \end{code}

% Ivory can interact with externally defined C functions and global
% variables. The \cd{importProc} primitive allows the user to declare an external
% procedure, and ensures the correct header file is included by the generated
% code.

% Ivory can only import and use functions that have a valid Ivory type signature.
% Some polymorphic C functions may have multiple valid Ivory types.
% \jamey{The printf\_proc example illustrates this, but does it pass the sanity
% checker? I don't think so...?}

% \subsection{Toolchain Use}
% The Ivory compiler is a Haskell function that takes a list of \cd{Module}s,
% parses command line options, and writes generated C source and header files to
% a directory given by those options.

% The compiler's second argument is a list of \cd{Artifact}s. Artifacts are a
% datatype for an arbitrary Haskell string and a filename, indicating the contents
% of a non-Ivory-generated file to be written to the output directory. In
% practice, this is used to write Makefiles, native C sources, linker scripts, and debug output
% specified by the user.

% There are also related functions exposed to the user that allow the parsing
% of command line options to be separated from the compile step, where desired.

% \begin{code}
% import Ivory.Compile.C (compile)

% main :: IO ()
% main = compile [ ex_module, print_ex_module ] []
% \end{code}







\section{Ivory Embedding}
\label{sec:ivory-embedding}

\lee{This section will be the heart of the paper, focusing on how we have
  embedded Ivory in Haskell's type system. I expect this to be 3-5 pages and
  probably the most technical. We want to be detailed, but describe things
  general enough that ML, Agda, etc. programmers can understand.
}

\subsection{The Ivory Monad}

\lee{Lee: describe the Ivory monad (not too much to say), then describe
  effects. Make sure to make clear these are monad effects like are popular in
  the literature. Mention inspiration from
  \url{http://www.doc.ic.ac.uk/~wlj05/files/Deconstraining.pdf}, but didn't
  follow implementation.}

\lee{Also talk about the use of typeclasses \cd{IvoryVar}, \cd{IvoryExpr},
  etc. and how this allows the user to define custom data, safely.}

Ivory is a monadic language, in which Ivory statements have effects in the Ivory
monad. The Ivory monad has the type

\begin{code}
Ivory (eff :: Effects) a
\end{code}

\noindent
and is a wrapper for a writer monad transformer over a state monad. The writer
monad writes statements into the Ivory abstract syntax tree, and the
state monad is used to generate fresh names for variables.

\paragraph{Effects}
The \cd{eff} type parameter of the Ivory monad is a phantom type that tracks
effects in the Ivory monad at the type level. (These effects have no relation to
the recent work on effects systems for monad transformers~\cite{}.) Currently,
we track three classes of effects for a given monadic code block:

\begin{itemize}
\item \emph{Returns}: does the code block contain a \cd{return} statement?
\item \emph{Breaks}: does the code block contain a \cd{break} statement?
\item \emph{Allocation}: does the code block contain local memory allocation?
\end{itemize}

Intuitively, these effects matter because their safety depends on the context in
which the monad is used. For example, a \cd{return} statement is safe when used
within a procedure, to implement a function return. But an Ivory code block can
also be used to implement an operating system task that should never
return. Similarly in Ivory, \cd{break} statements are used to terminate
execution of an enclosing loop (the other valid use of \cd{break} in C99 is to
terminate execution in a \cd{switch} block; Ivory does not contain
\cd{switch}). By tracking break effects, we can ensure that an Ivory block
containing a \cd{break} statement is not used outside of a loop. Finally,
allocation effects are used to guarantee that a reference to locally-allocated
memory is not returned by a procedure, resulting in undefined behavior; see
Section~\ref{sec:ref} for details. Moreover, from a code block's type alone, we
can determine whether it allocates memory, simplifying tasks like stack usage
analysis.

The Ivory effects system is essentially implemented by a type-level tuple
structure where each of the three effects correspond to a field of the
tuple. Type equality constrains enforce that a particular effect is (or is not)
allowed in a give function signature.

The type-level tuple is implemented using data kinds~\cite{}. Effects have the
kind \cd{Effects}, containing a single type constructor, \cd{Effects}. The type
constructor \cd{Effects} is parameterized by the fields of the tuple,
representing the respective effects. Using GHC's data kinds extension to lift
data type declarations to data kind declarations,
\begin{code}
data Effects = Effects ReturnEff BreakEff AllocEff
\end{code}

\noindent
Consider the return effects type, \cd{ReturnEff}.  Again using the data kinds
extension to define a new kind and its types, we define two types denoting
whether returns are permitted or not.

\begin{code}
data ReturnEff = forall t. Returns t | NoReturn
\end{code}

\noindent
The type is existentially quantified, since in the case of permitted returns, we
parameterize the type constructor with the type of the value being returned.

A type family~\cite{} is used to access and modify the types at each field of
the tuple. To access that field, a type family rewrites the \cd{Effects} type to
the return effect type \cd{ReturnEff}.

\begin{code}
type family   GetReturn (effs :: Effects) :: ReturnEff
type instance GetReturn ('Effects r b a) = r
\end{code}

\noindent
(The GHC convention is to precede a type with a tick (\cd{'}) to disambiguate a
type constructor (with the tick) from the data constructor.)

To modify the field, another type family rewrites it. For example,

\begin{code}
type family   ClearReturn (effs :: Effects) :: Effects
type instance ClearReturn ('Effects r b a) =
  'Effects 'NoReturn b a
\end{code}

With this machinery, we can now use a type equality constraint to enforce
particular effects in a context. For example, the type of \cd{ret}, the smart
constructor for returning a value from a procedure, has the following type
(additional type constraints are elided):

\begin{code}
ret :: (GetReturn eff ~ Returns r, ...)
    => r -> Ivory eff ()
\end{code}


\subsection{References and Allocation}
\label{sec:ref}
Ivory manages allocated data through the use of non-nullable references.
References are represented using the \cd{Ref} type, which takes two parameters:
the scope that it was allocated in, and the type of the memory area it points
to.

All references in Ivory are allocated in one of two scopes, \cd{Global}, or a
fresh local scope, unique to the function that the allocation takes place
within.  References with \cd{Global} scope are allocated through the use of the
\cd{area} top-level declaration, and references allocated within a function are
allocated through the use of the \cd{local} function.  Both can take
initializers, defaulting to zero-initialization when they are omitted.

While references allocated in the \cd{Global} scope are allocated when the
program starts, references allocated in the local scope of a function are stack
allocated, and freed as soon as the function returns.

\subsection{Memory Areas}
In Ivory, data pointed to by references and pointers is described by the {\tt
Area} kind, following Diatchki and Jones' work in \cite{memareas}.  This kind
contains four types: {\tt Stored}, \cd{Struct}, \cd{Array}, and \cd{CArray}.

\begin{figure}[h]
\begin{code}
data Area k = Array Nat (Area k)
            | CArray (Area k)
            | Struct Symbol
            | Stored k
\end{code}
\caption{The definition of the \cd{Area} kind}
\end{figure}

\paragraph{Stored values}
The simplest type of memory area is a single base type, lifted to the \cd{Area}
kind by the use of the \cd{Stored} type constructor.  For example, the area type
of a \cd{Sint32} would simply be \cd{Stored Sint32}.

\paragraph{Structs} A reference that has an area-kind of type \cd{Struct "x"}
will point to memory whose layout corresponds to the definition of the struct
with name x.  Struct definitions are introduced through use of the ivory
quasi-quoter.  For example, if a region of memory is typed using the struct
definition from figure \ref{example-struct}, it would have type \cd{Struct "a"}.

\begin{figure}[h]
\begin{code}
[ivory| struct a { field1 :: Stored Sint32
                 , field2 :: Struct "b"
                 }
      |]
\end{code}
\caption{An example struct definition}
\label{example-struct}
\end{figure}

Also introduced by the struct declaration are field labels.  Field labels allow
for indexing into a memory area, producing a reference to the value contained
within the struct.  For example, using the struct definition from figure
\ref{example-struct}, there would be two labels introduced: \cd{field1}, and
\cd{field2}, and will have types \cd{Label "a" (Stored Sint32)} and \cd{Label
"a" (Struct "b")}, respectively.

Using a struct label to select the field of a structure requires the use of the
\cd{(\mytilde>)} operator, which expects a reference to a structure as its first
argument, and a compatible label as its second.  In figure
\ref{example-struct-label}, the \cd{(\mytilde>)} operator is used with a reference to
an ``a'' struct, with the \cd{field1} label, producing a new reference of type
\cd{Ref Global (Stored Sint32)}

\begin{figure}[h]
\begin{code}
example :: Ref Global (Struct "a")
        -> Ref Global (Stored Sint32)
example ref = ref %*\mytilde*)> field1
\end{code}
\caption{Struct field indexing}
\label{example-struct-label}
\end{figure}

\paragraph{Arrays}
Arrays in Ivory take two type parameters: the length of the array as a
type-level natural number, and the area type of its elements.  For example, an
array of 10 \cd{Sint32} would have the type \cd{Array 10 (Stored
Sint32)}.  Indexing into arrays is accomplished through the use of the \cd{(!)}
operator, shown in figure \ref{array-support-functions}.

An index into an array has the type \cd{Ix}, which is parameterized by the size
of the array that it is indexing into.  The \cd{Ix n} type will only hold
values between zero and \cd{n-1}, which allows us to avoid run-time array bounds
checks, as in \ref{memareas}.  One shortcoming of this approach is that the
\cd{(!)} operator will only accept indexes that are parameterized by the length
of the array being indexed, while it would be useful to allow indexes that have
a maximum value that is less than the length of the target array.

As array indexes are parameterized by the length of arrays they can index into,
they become an interesting target for new combinators.  In this vein, we
introduce \cd{arrayMap}, whose signature is shown in figure
\ref{array-support-functions}.  The intuition for the \cd{arrayMap} function is
that it invokes the function provided for all indexes that lie between $0$ and
$n - 1$.  As the index argument given to the function is most often used with an
array, type information propagates out from uses of the \cd{(!)} operator, and
it becomes unnecessary to give explicit bounds for the iteration.

\begin{figure}[h]
\begin{code}
(!)      :: Ref s (Array n area) -> Ix n -> Ref s area
arrayMap :: (Ix n -> Ivory eff a) -> Ivory eff a
toCArray :: Ref s (Array n area) -> Ref s (CArray area)
arrayLen :: Num len => Ref s (Array n area) -> len
\end{code}
\caption{Array support functions}
\label{array-support-functions}
\end{figure}

For compatibility with C, we also introduce a type for arrays that are not
parameterized by their length, \cd{CArray}.  There are no operations to work
with references to \cd{CArray}s in Ivory, as the assumption is that they will
only ever be used when interacting with external C functions.  As many C
functions that consume arrays require both a pointer and a length, we also
provide the \cd{arrayLen} function, which allows the length of an Ivory array to
be demoted to a value.  When used in conjunction with \cd{toCArray}, this
function allows for fairly seamless integration with external C code.

\subsection{Procedures}
Procedures in Ivory inhabit the \cd{Def} type which is parameterized by the
signature of the function it names.  Procedure signatures inhabit the \cd{Proc}
kind, which provides one type constructor: \cd{:->}.  The \cd{:->} type
constructor takes two arguments: the types of the argument list, and the return
type of the whole procedure.  To enable processing using type classes while
easing the burden on the programmer, the argument list is specified as a
type-level list of $*$-kinded types.  This enables the definition of three
classes that are used in the definition and invocation of procedures.

\paragraph{Definition} Procedures are defined through the use of the \cd{proc}
function, which requires two arguments: a symbolic name for the function, and
its implementation.  The implementation takes the form of a Haskell function
that operates over Ivory language values, producing a function body result.
Correct procedure definition is guarded by the \cd{IvoryProcDef} class, which
constrains the use of the \cd{proc} function.

\cd{IvoryProcDef} has two parameters: signature and implementation, which relate
the \cd{proc} type of the Ivory procedure and the haskell function given as the
its implementation.  There are only two instances for \cd{IvoryProcDef}: the
case where the argument list is empty, and the case where the argument list is
extended by one argument.  The latter case also requires that the argument added
be an Ivory type (via the \cd{IvoryExpr} class), thus guaranteeing that values
can be generated to pass to the implementation function.   As the signature of
the resulting \cd{Def} is fully determined by the implementation function
through the functional dependency from impl to proc in the class definition of
\cd{IvoryProcDef}, users are only required to write signatures when the
implementation given to \cd{proc} is ambiguous.

The implementation function is required to produce a final value of the type
\cd{Body r}, where \cd{r} is the return type of the function.  Values of type
\cd{Body} are created through the use of three functions: \cd{requires},
\cd{ensures}, and \cd{body}.  The \cd{requires} function allows the programmer
to state requirements of the arguments to the function, while \cd{ensures}
allows the programmer to state properties that the function should preserve. The
third function \cd{body}, whose signature is shown in figure \ref{proc-defs},
lifts an Ivory computation that returns a result \cd{r} and allocates data in a
region \cd{s} into a value of type \cd{Body r}.  As the allocation scope
expected by the Ivory computation given is quantified in a rank-2 context to the
\cd{body} function, it is not permitted to show up in the type of the result,
\cd{r}.  This behavior allows us to prevent anything allocated within the
implementation function from being returned, a source of dangling pointer bugs.
This is the same technique used by Launchbury and Peyton Jones in \cite{stmonad}
to prevent mutable state from leaking out of the context of the run function for
the \cd{ST} monad.

For example, the procedure \cd{f} defined in figure \ref{proc-def}, will produce
a type error, as it attempts to return a locally-allocated reference; as
references have the scope they were allocated in as part of their type, the use
of the \cd{Body} type prevents anything allocated within its context from
escaping out.

\begin{figure}[h]
\begin{code}
f = proc "f" $ body $ do
  ref <- local (izero :: Init Sint32)
  ret ref
\end{code}
\caption{Attempted creation of a dangling pointer}
\label{proc-def}
\end{figure}

\paragraph{Invocation} Procedures are invoked directly through the use of the
\cd{call} function or indirectly through the \cd{invoke} function, both of which
take a \cd{Def} as their first argument, using its signature to determine the
rest of the arguments needed.  The arguments needed are determined by the
\cd{IvoryCall} class, which uses the signature information to produce a
continuation that requires parameters that match the type of the argument list
from the signature of the \cd{Def}.  The \cd{IvoryCall} class mirrors the
structure of the \cd{IvoryProcDef} class in instance structure, though it adds
one additional parameter: \cd{eff}.  This additional parameter is required so
that the containing effect context of the call can be connected to the result of
the continuation generated by the instances of \cd{IvoryCall}.  For example,
calling a procedure with type \cd{Def ('[Sint32] :-> Sint32)} will produce a
continuation of the type, \cd{Sint32 -> Ivory eff Sint32}, where the \cd{eff}
parameter is inherited from the current environment.

\trevor{Is there something from Lennart that we can cite about the
implementation of call, as it relates to printf?}

\begin{figure}[h]
\begin{code}
data Proc k = [k] :-> k

class IvoryProcDef (sig :: Proc *) impl | impl -> sig
instance IvoryProcDef ('[] :-> r) (Body r)
instance IvoryProcDef (as :-> r) impl
  => IvoryProcDef ((a ': as) :-> r) (a -> impl)

class IvoryCall eff (sig :: Proc *) impl
  | sig eff -> impl, impl -> eff
instance IvoryCall eff ([] :-> r) (Ivory eff r)
instance (IvoryExpr a, IvoryCall eff (as :-> r) impl)
  => IvoryCall eff ((a ': as) :-> r) (a -> impl)

body :: (forall s. Ivory (ProcEffects s r) () -> Body r

data Def (sig :: Proc *)
proc :: IvoryProcDef sig impl
     => Sym -> impl -> Def sig

call :: IvoryCall sig eff impl => Def sig -> impl
\end{code}
\caption{Function definition support}
\label{proc-defs}
\end{figure}



\lee{Trevor: describe the use of type-level lists to define procedures. Describe
the use of a type-class to ensure procedure calls in Ivory are type-correct.}

\subsection{Strings}
Ivory's support for character strings builds upon the fixed-size array
data type to implement an "array with fill pointer" mechanism, as opposed
to traditional C-style null-terminated strings.

A string value is a structure containing two fields: the data array
with its type-level natural number parameter indicating the maximum
capacity of the string, and an integer field containing the number of
valid characters in the string, up to the capacity.

In figure \ref{ivory-string-type-defn}, we define two example
string types: \cd{Name} is defined with a capacity of 40
characters, and \cd{Phone} with a capacity of 10 characters.

\begin{figure}[h]
\begin{code}
[ivory| struct name
    { name_data   :: Array 40 (Stored Uint8)
    ; name_length :: Stored Sint32
    }

  struct phone
    { phone_data   :: Array 10 (Stored Uint8)
    ; phone_length :: Stored Sint32
    }
|]

type Name  = Struct "name"
type Phone = Struct "phone"
\end{code}
\caption{Definition of two Ivory string types}
\label{ivory-string-type-defn}
\end{figure}

Despite their common structure, the \cd{Name} and \cd{Phone} types
are completely distinct. In order to provide standard library
functions that operate over strings with different capacities in a generic
way, we capture this common structure in the \cd{IvoryString} type
class, defined here in figure \ref{ivory-string-class}.

\begin{figure}[h]
\begin{code}
class IvoryString a where
  type Capacity a :: Nat
  stringDataL   :: Label (StructName a)
                         (Array (Capacity a)
                                (Stored Uint8))
  stringLengthL :: Label (StructName a)
                         (Stored Sint32)
\end{code}
\caption{The \cd{IvoryString} type class}
\label{ivory-string-class}
\end{figure}

The \cd{IvoryString} type class ties together the type-level natural
number containing the array's capacity % (using an associated type \cite{})
along with structure field accessors
for the data array and length field. The instance declarations for \cd{Name}
and \cd{Phone} are trivial, as shown in figure \ref{ivory-string-instances}.

\begin{figure}[h]
\begin{code}
instance IvoryString Name where
  type Capacity Name = 40
  stringDataL = name_data
  stringLengthL = name_length

instance IvoryString Phone where
  type Capacity Phone = 10
  stringDataL = phone_data
  stringLengthL = phone_length
\end{code}
\caption{\cd{IvoryString} instances for the \cd{Name} and \cd{Phone}
string types.}
\label{ivory-string-instances}
\end{figure}

With these instances in place, we are able to provide standard library
functions that operate on strings of any capacity, without losing any
type-level information about the capacity of the internal array. For
example, to return the length of any Ivory string, we define the
\cd{istr\_len} function as follows:

\begin{code}
-- Return the length of any Ivory string.
istr_len :: IvoryString a
         => ConstRef s a -> Ivory eff Sint32
\end{code}

The process of defining string types that only differ
in their maximum capacity can be automated by the Ivory quasiquoter,
allowing us to define these types in a single line each:

\begin{code}
[ivory|
  string struct Name 40
  string struct Phone 10
|]
\end{code}

\subsection{Bit-Data}
\lee{James: talk about using TH to define bitdata.}

\subsection{Maybe Types}
\lee{James: not sure, but maybe this would be useful to have?}

\subsection{Module System}
\lee{Lee: describe the module system, warts and all.}

As seen in Section~\ref{sec:ivory-overview}, Ivory's module system packages up
the collection of procedures, data declarations, etc. to be passed to a
back-end, such as the C code generator. The module system is implemented simply
as a writer monad that produces a list of abstract syntax values that are
processed by the various back-ends.

While the module system implementation is intuitive and simple, it is also one
of the greatest weaknesses of Ivory as an EDSL. \lee{XXX finish}

Values included in a module must be monomorphic; a type error results
otherwise. 

Because of our design decision 

\lee{make sure Pat's section show the Ivory module system.}








\newcommand{\coreivory}{Core Ivory}

\section{Ivory Semantics}
\label{sec:semantics}

In the previous section, we described our embedding of Ivory into the
GHC type system and made the claim that this guarantees memory safety.
We modeled a simplified version of the Ivory language inside
Isabelle/HOL\cite{isabelle}, henceforth \emph{\coreivory{}}, to support
this claim.  In this section we present a semantics based upon the
Isabelle/HOL development, and outline the proof of type safety.

Developing this model provides a number of benefits for a modest
investment---we developed the model in under a person month, albeit
one of the authors has significant experience with Isabelle.  In
addition to the basic benefits formalisation provides, we can
experiment with extensions to Ivory.

In one such experiment, we extended the model to allow references in
the heap, a feature we avoided in the development of Ivory due to
soundness concerns.  While a simple extension to the syntax and
semantics of Ivory, the effort involved in extending the soundness
proofs was almost as much as developing the initial model.  

Due to space constraints we discuss only those particulars of
\coreivory{} which differ significantly from a standard imperative
\trevor{Remember to fix this reference in the final version}
language; see the supplemental material for more details.

\newcommand{\sep}{\ |\ }
\newcommand{\syntaxtitle}[1]{\multicolumn{3}{l}{\textit{#1}}}

\begin{figure}[t]

\[
\begin{array}{crl}
\syntaxtitle{pure expressions}\\
e & ::= & 0 \sep{} 1 \sep{} \ldots{} \sep{} \texttt{true} \sep{} \texttt{false} \sep{} () \sep{} x \sep{} 
          e_1 \mathbin{\mathit{op}} e_2 \\
\syntaxtitle{impure expressions}\\
i & ::= & \texttt{pure}(e) \sep{} \texttt{alloc}(e) \sep{}
          \texttt{read}(e) \sep{} \texttt{write}(e_1, e_2)\\
\syntaxtitle{statements}\\
s & ::= & \texttt{skip} \sep{} \texttt{return}(e) \sep{} s_1; s_2 \\ 
  & |   & \texttt{if}(e)\texttt{\;then\;} s_1 \texttt{\;else\;} s_2 \\
  & |   & \texttt{for}(x = e_1; e_2; e_3) \{ s \} \\
  & |   & \texttt{let\;} x = i \texttt{\;in\;} s \\
  & |   & \texttt{let\;} x = f(e_1, \ldots{}, e_n) \texttt{\;in\;} s \\
\syntaxtitle{values}\\
v & ::= & 0 \sep{} 1 \sep{} \ldots{} \sep{} \texttt{true} \sep{} \texttt{false} \sep{} () \\
w & ::= & \texttt{stored}(v) \sep{} \texttt{ref}(r, n) \\
\syntaxtitle{Stores}\\
E & \in{} & x \to w \\
\syntaxtitle{regions and heaps}\\
R & \in{} & \mathbb{N} \to w\\
H & ::=   & H, R \sep{} \emptyset{}\\
\syntaxtitle{stacks}\\
F & ::= & \texttt{rframe}(x, E, s) \sep{} \texttt{sframe}(E, s) \\
S & ::= & F, S \sep{} \emptyset{} \\
\syntaxtitle{configurations}\\
C & \in & H \times S \times E \times s \sep{} \texttt{finished}(v) \\
\syntaxtitle{types}\\
\rho   & \in & \textit{region variables}\\
\alpha & ::= & \texttt{nat} \sep{} \texttt{bool} \sep{} \texttt{unit} \\
\tau   & ::= & \texttt{storedt}(\alpha) \sep{} \texttt{reft}(\rho, \alpha) \\
\syntaxtitle{procedure definitions}\\
P & ::= & \texttt{proc\;} f(\tau_1\;x_1, \ldots{}, \tau_n\;x_n) : \tau\; \{ s \}
\end{array}
\]

\sjw{values (in or out?), $\in$ or $\subseteq$?}

\label{fig:syntax}
\caption{Concrete syntax of \coreivory{}}
\end{figure}

\subsection{Syntax}

The syntax for \coreivory{} is given in \autoref{fig:syntax}.
\coreivory{} is based upon a typical typed imperative language with
function calls, references, and memory allocation (but not memory
deallocation).  \coreivory{} attempts to stay faithful to Ivory
wherever possible, and so variables are let-bound with forms for
binding the result of expression evaluation and function
calls. Furthermore, \coreivory{} expressions are stratified into
\emph{pure} and \emph{impure}, the latter allowing operations on the
heap: allocation, reading, and writing references.  

Ivory uses regions to manage memory.  Thus, the heap is modeled as a
list of \emph{regions}, each region a finite map from \emph{offsets},
modeled as natural numbers, to \emph{stored values}; Ivory does not
allow references in heap allocated values, and so a stored value is
any value which is not a reference. A \emph{reference} contains both a
\emph{region index} into the list of regions, and an offset with the
region.  To simplify the presentation, we will use $H(r, n)$ to denote
the value at offset $n$ in the $r$th region of $H$, and similarly with
updates.

As with values, types classifying values in \coreivory{} are
stratified into storable and reference types; a reference type
$\texttt{reft}(\rho, \alpha)$ is a reference to an object of type
$\alpha$ in region $\rho$, where $\alpha$ is not a reference.

\sjw{talk about structs and arrays?}

\subsection{Operational Semantics}

% These are here so we can talk about them in the text, but they are
% mainly used in the appendix

\newcommand{\stepsX}[2]{\models #1 \longmapsto{} #2}
\newcommand{\stepsXX}[3]{\models #2 \longmapsto^{#1} #3}
\newcommand{\steps}[4]{\stepsX{#1; #2}{#3; #4}}

\newcommand{\denoteexp}[2]{\llbracket{}#1\rrbracket{}#2}

\newcommand{\stepsH}[2]{#1 \longmapsto_I #2}
\newcommand{\hsteps}[5]{#1 \models \stepsH{#2; #3}{#4; #5}}

% We ignore the procedure environment for clarity
\newcommand{\wfstmt}[5]{#1; #3 \vdash_s #4 : #5}
\newcommand{\wfexp}[3]{#1 \vdash_e #2 : #3}
\newcommand{\wfimp}[4]{#1; #2 \vdash_i #3 : #4}

\newcommand{\rulestitle}[1]{\textnormal{#1}\hspace*{\fill}}

\coreivory{}'s semantics are modeled as an abstract machine
over configurations.  The judgement
\[
\stepsX{C}{C'}
\]
states that configuration $C$ transitions to configuration $C'$.  A
configuration consists of a heap, a stack, a store, and the current
statement. The stack contains continuations for both function calls
and statement sequences while the store maps variables to values. The
semantics of sequencing is slightly non-standard as variables are
let-bound rather than assigned, and so statement sequencing preserves
the store across execution of the first statement.

For example, the semantics of dereferencing is given by the rule
\begin{mathpar}
\inferrule{ \denoteexp{e}{E} = \texttt{ref}(r, n)  \\ (r, n) \in \texttt{dom}(H) }%
          {\steps{(H; S; E)}{\texttt{let\;} x = \texttt{read}(e) \texttt{\;in\;} s}{(H; S; E[x \mapsto{} w])}{s}}
\end{mathpar}
where \( \denoteexp{e}{E} \)
evaluates the pure expression $e$ under the store $E$.  The premise
\( (r, n) \in \texttt{dom}(H) \)
requires the existence of the region and offset pointed to by the
reference resulting from evaluating $e$.

Operationally, the heap is extended on a function call, an empty
region being added to the end of the list, and shrunk on function
return, removing the last region.  Allocating an object extends the
current (last) region.

A configuration is stuck if there is no available transition.  For
instance, an attempted heap access or update where the region index
does not exist or at an offset which has not been allocated will
result in a stuck configuration.  In particular, accessing a region
after it has been removed will result in a stuck state.  

\subsection{Typing Ivory}

\coreivory{}'s typing judgements extend standard statement typing
judgements with a current region variable.  The typing judgement 
\[
\wfstmt{\Gamma}{\Psi}{\rho}{s}{\tau}
\]
holds when the statement $s$ is well-formed under the store
environment $\Gamma$, current region $\rho$, with any return
statements returning values of type $\tau$.

The region variable $\rho$ represents the current region and is used
when checking memory allocation.  The typing rule for allocation is then
\[
\inferrule{ 
\wfexp{\Gamma}{e}{\alpha} \\
\wfstmt{\Gamma[x \mapsto \texttt{reft}(\rho, \alpha)]}{\Psi}{\rho}{s}{\tau} }%
{ \wfstmt{\Gamma}{\Psi}{\rho}{\texttt{let\;} x = \texttt{alloc}(e) \texttt{\;in\;} s}{\tau} }
\]
where the body of the let statement is checked under the additional
assumption that the variable $x$ has reference type
$\texttt{reft}(\rho, \alpha)$, noting the region variable on the
reference type comes from the current region variable.  The judgement 
\( \wfexp{\Gamma}{e}{\alpha} \) holds when pure expression $e$ has type $\alpha$.

We fix the set of procedures as \texttt{Procs}. The typing rule for procedure bodies
\begin{mathpar}
\inferrule{\forall\;\texttt{proc\;} f(\tau_1\;x_1, \ldots, \tau_n\;x_n) : \tau\;\{ \mathit{body} \} \in \texttt{Procs}\\\\
            \rho \textrm{ fresh} \\     
            \texttt{frees}(\tau) \subseteq \texttt{frees}(\tau_1) \cup \ldots \cup \texttt{frees}(\tau_n) \\
            \wfstmt{[x_1 \mapsto \tau_1, \ldots, x_n \mapsto \tau_n]}{\Psi}{\rho}{s}{\tau} }%
{ \vdash \texttt{Procs} }
\end{mathpar}
ensures that this region variable
is fresh; this constraint, together with the constraint that region
variables in the procedure's return type must occur in an argument
type, ensures that references cannot escape the scope in which they
were allocated.  These constraints are then fundamental to the type
safety of \coreivory{} programs.

\subsection{Type Safety}

We prove type safety, that is, well-typed programs do not get stuck,
by proving the usual progress and preservation lemmas.  As is
common  with type safety for imperative languages, we
define auxiliary well-formedness invariants on configurations.  The
progress lemma then states that well-formed configurations are not
stuck, and preservation states that well-formed configurations
transition to well-formed configurations.

\begin{theorem}[Type Safety]
Given \( \vdash \texttt{Procs} \) and
\[
\texttt{proc\;} \texttt{main}() : \texttt{nat}\;\{ \mathit{s} \} \in \texttt{Procs}\\
\]
there either exists some number of steps $n$ and value $v \in \mathbb{N}$ such that
\[
\stepsXX{n}{(\emptyset; \emptyset; \emptyset); \texttt{let\;} x = \texttt{main}() \texttt{\;in\;} \texttt{return}(x)}{\texttt{finished}(v)}
\]
or, for all $n$ there is a well-formed configuration $C$ such that
\[
\stepsXX{n}{(\emptyset; \emptyset; \emptyset); \texttt{let\;} x = \texttt{main}() \texttt{\;in\;} \texttt{return}(x)}{C}
\]
\end{theorem}
Informally, a well-formed program, when called via the \texttt{main}
procedure, will either terminate in a finite number of steps, or will
diverge through well-formed configurations.


The well-formedness invariants are typical, and are based upon a
well-formed value judgement.  For our purposes, the interesting rule
here is for references
\[
\inferrule{\Delta(\rho) = r \\ \Theta(r, n) = \alpha }%
{\Theta; \Delta \vdash \texttt{ref}(r, n) : \texttt{reft}(\rho, \alpha)}
\]
which links reference types to reference values through the region
environment $\Delta$, mapping region variables to region indices, and
heap type $\Theta$, mapping region indices and offsets to types.
Well-formed heaps and stores then follow point-wise from this
judgement.  Well-formedness of stacks follows, ensuring that each
continuation on the stack is well-formed.

A well-formed configuration, in addition to the well-formedness of the
heap, stack, store, and current statement, constrains
the region environment ($\Delta$).  In particular, the variable
representing the current region must be mapped to the length of the
current heap, every region index in the range of $\Delta$ must be
below this length, and the type variables occurring in the store
environment $\Gamma$ must be mapped by $\Delta$.

The progress lemma follows from well-formedness.  The preservation
proof, as usual, is the trickier of the two proofs.  In particular,
the case for return involves showing that the various configuration
members are well-formed under a heap where the last region has been
removed from the heap.  This involves showing that references are
well-formed under this smaller heap which follows from the stack and
region environment well-formedness invariants.  The case for function
calls is also involved, although this is primarily due to the
instantiation of the type variables in the type of the called
function.

\subsection{Discussion}
\label{sec:sem-discuss}

The type safety proofs are greatly simplified by the restriction of
heap values to non-references: it is trivially true, for example, that
the heap is well-formed after a return as the well-formedness of
non-reference values does not depend on the type of other heap
elements.

We extended the Isabelle model to remove the stratification of values,
allowing references to appear in the heap.  Syntactically, this
simplifies the language, at the expense of more complicated proofs.
For example, we extend the well-formedness of heaps to require that
heaps are downward closed with respect to region references: that is,
references in the heap refer only to regions up to and including the
containing region.  This work extended the proof development from
approximately 2500 lines of proof to 3300 lines of proof, and took
approximately 2 person weeks to finish.

\sjw{move to end?}
Developing this model uncovered a bug in the Ivory embedding into
Haskell, namely that the Haskell erroneously allowed functions to
end without encountering a return statement.  Thus, a program could
claim to return, say, a valid index but in reality return an
out-of-bounds index, breaking memory safety.  In the model this
check is part of the type checking rules (although elided for clarity
in this paper) but is implemented as an explicit type-check over the Ivory AST
in the
Haskell implementation.  

The existence of the formal model is not a guarantee that the Ivory
implementation is sound, however: there is no formal link between the
implementation and the model.  For example, we discovered a number of
bugs due to the way in which references are initialized: these bugs
are impossible in the model, but allowed in the implementation.  

In addition, \coreivory{} covers a subset of the Ivory language,
currently missing data structures and arrays.  We are working to
reduce this gap; furthermore, in future, we plan to investigate making
the core of the implementation more closely resemble \coreivory{}.
Doing so would have the additional benefit of allowing us to verify
the correctness of operations performed over the core AST.

We plan on investigating first-class regions as a further extension to
the model.  This feature would allow programmers to name and pass
around allocation contexts, allowing helper functions to, for example,
allocate and return objects in a parent functions region.  While we
are confident that this extension is sound, \emph{proving} it sound
allows a much greater confidence in the implementation: memory
deallocation is notoriously easy to get wrong, so having a formal
proof of soundness would be greatly comforting.


\section{Ivory Testing and Verification}
\label{sec:tools}

Ivory contains built-in tools to support high-assurance software
development. These tools include a correctness condition generator, a
SMT-based symbolic simulator, a theorem-prover translator, and a
QuickCheck\cite{qc} engine for randomized testing. We describe each of
them briefly below.

% \sjw{er, what?}
% None of these tools specifically depend on Ivory being an EDSL, and their
% implementations are straight-forward.

\subsection{Correctness Conditions}
Some correctness properties such as arithmetic underflow and overflow
cannot be embedded in the Haskell type system. For this class of
properties the translation into the Ivory AST adds instrumentation
containing appropriate assertions. For example, from the function

\begin{code}
add_ex :: Def ('[Sint32, Sint32] :-> Sint32)
add_ex = proc "add_ex" $ \x y -> body $
  ret (x + y)
\end{code}
\noindent
The following function is generated
\begin{code}
int32_t add_ex(int32_t var0, int32_t var1)
{
    bool i_ovf0 = add_ovf_i32(var0, var1);
    COMPILER_ASSERTS(i_ovf0);
    return (int32_t) (var0 + var1);
}
\end{code}
\noindent
Note the addition of the an overflow check, performed by the function
\cd{add\_ovf\_i32}, before the expression is evaluated. The result of
this check is handled by the platform-dependent macro
\cd{COMPILER\_ASSERTS}; the response when a property is
violated is platform-dependent, and may include logging, do-nothing, run a
recovery procedure, and so forth.

Ivory inserts correctness condition checks for the following conditions, as requested by
the user:
\begin{itemize}
\item no arithmetic underflow and overflow,
\item no division-by-zero,
\item no bit-shifts are greater than or equal to the value's width,
  (bit-shifts on signed integers are prevented statically by the type system)
\item no floating-point operations result in \cd{inf} or \cd{NaN} values.
\end{itemize}

The implementation is mostly straightforward; the two aspects that require some
care of are sharing and control-flow expressions. As an EDSL, Ivory encourages
the use of macros which can result in large expressions. In this case, each
subexpression must be checked. Standard common subexpression elimination can
dramatically reduce the number of instrumented assertions. Second, Ivory contains
the short-cutting expressions of conjunctions, disjunctions, and conditionals,
analogs of C's \cd{\&\&}, \cd{||}, and \cd{\_ ? \_:\_}, respectively. For these
expressions, the generated assertions must contain as a precondition that
short-cutting has not occurred so as not to be overly pessimistic. For example,
for the expression
\noindent
\begin{code}
x != 0 ? 3/x : 0
\end{code}
\noindent
the (tautological) assertion
\begin{code}
(x == 0) || (x != 0)
\end{code}
\noindent
is generated.

Generated correctness conditions as well as general user assertions can be
discharged via testing or using model-checking or theorem-proving. We describe
Ivory's tooling for these approaches below.

\paragraph{Symbolic Simulation}
Ivory contains a symbolic simulator built over CVC4~\cite{cvc4} for verifying
programs. Ivory programs are typically amenable to formal analysis since they
guarantee the absence of memory-safety errors, there is no pointer arithmetic,
there is no heap, and they are not concurrent. Given a set of preconditions, the
simulator attempts to verify any inline assertions and postconditions. The
simulator ensures that the collection of preconditions are satisfiable.

The symbolic simulator abstracts various domains. Floating point types are
abstracted as reals. However, fixed-width values are modeled precisely as are
arrays. We have not yet incorporated a theory of bit-vectors yet,
however. \eric{The last two sentences may be a bit confusing, since you'd usually expect a precise encoding of fixed-width integers to use bit-vectors. Of course we get away with using integers since we strictly disallow overflow. Perhaps we should clarify here?} Non-linear operators are abstracted with linear contracts. Loops are
unrolled (all loops in Ivory have a constant upper bound).

During analysis, inter-procedural calls are inlined or abstracted based a on
user-supplied option. If they are abstracted, then the callee's precondition is
added as a verification condition at the call site. The callee's postcondition
is added to the set of invariants following the call. For imported procedures
hand-written in C, Ivory allows them to be augmented with pre- and
postconditions to abstract them.

The simulator is simple, providing only bounded model-checking, no support for
concurrency, and no counter-example guided refinement, matching the simplicity
of the language.

We have used the symbolic simulator to analyze various Ivory programs, having
found an off-by-one bug in a ring buffer and verifying the correctness of a
safety state-machine in SMACCMPilot~\cite{smaccm}.

\paragraph{Theorem Proving}

Eakman~et~al. has implemented a theorem-prover back-end for
Ivory~\cite{ivory-acl2} that targets ACL2~\cite{acl}. The theorem-prover
performs inter-procedural analysis, abstracting procedure calls by their
contracts, like the symbolic simulator. However, the theorem prover has the
ability to scale more substantially, at the price of doing interactive
proofs. Eakman~et~al. provide examples of the use of the theorem-proving
back-end.

\paragraph{Property-Based Testing}

Finally, we have implemented a QuickCheck~\cite{qc} like property-based testing
framework for Ivory. The framework tests procedures by randomly generating
values both for their formal arguments as well as for global values referenced
by the procedure. Ivory is primarily a compiled language, so tests are compiled
to C to be executed. Only values that satisfy procedure preconditions should be
generated, so the framework contains an interpreter for evaluating values before
code generation.




\section{A Safe-C EDSL}
\label{sec:edsl}

Large-scale, safe~C programming in an EDSL has advantages and disadvantages. In
a previous experience report, we explored some of the benefits of using an EDSL
for embedded system development~\cite{smaccm}. We will not repeat those claims
here. Rather, we describe two problematic aspects we have specifically addressed in Ivory:
(1) integrating a C-like concrete syntax into the EDSL and (2) error-reporting.

\paragraph{Concrete Syntax}

A benefit of the EDSL approach is that it relieves the developer from having to
define and implement a front-end syntax. However, that also generally means that
only users of the host language will be attracted to using the
EDSL. We want C/C++ developers to use Ivory!

\begin{figure}[h!]
\begin{code}
void mapProc(s*uint8_t[4] arr, uint8_t x) {
  map ix {
    let v = arr@ix;
    *v = *v + x;
  }
}
\end{code}
  \caption{Concrete Syntax for Ivory}
  \label{fig:concrete}
\end{figure}

The need for a concrete C-like syntax became evident in our work supporting
Boeing's use of Ivory, mentioned in Section~\ref{sec:introduction}. We have
already seen specific uses of quasi-quotation in Ivory to define a concrete
syntax for structs (Section~\ref{sec:area}) and bit-data
(Section~\ref{sec:bitdata}). In fact, a quasi-quotation is given for the entire
Ivory language. An example quasi-quoted Ivory procedure is shown in
Figure~\ref{sec:control}. The procedure is the equivalent of the procedure with
the same name in Section~\ref{sec:control}.

A quasi-quoted Ivory program is guaranteed to be type-safe, since the generated
Haskell program is type-checked. an important feature of the quasi-quoted
language is that it automatically generates the appropriate type signatures for
Ivory programs, relieving the programmer from doing so. The quasi-quoter also
generates Ivory modules automatically and guarantees that procedure names match
their Haskell identifiers, obviating the problems discussed in
Section~\ref{sec:modules}. The quasi-quoter supports anti-quotation, so that
Haskell can still be used as a macro language. All of Boeing's development in
Ivory is via the quasi-quoter.

Implementing a quasi-quoter means that we have defined a lexer and parser for
the concrete syntax. Because Ivory is deeply embedded in Haskell, there is a
concrete data type (i.e., the abstract syntax tree (AST)) over which
optimizations and back-ends are implemented. The distance from Ivory as an EDSL
and a stand-alone compiler is surprisingly small, essentially requiring a
type-checker and a front-end targeting the AST directly. In this manner, we are
able to ``grow'' a compiler, from an EDSL to a stand-alone system.

\paragraph{Error Reporting}
Ivory's The use of advanced type-system features can produce elaborate and
confusing type-error messages to the user. Idris~\cite{christiansen2014reflect}
allows the programmer to supply error handlers, which are given a data structure

Statically, the use of advanced type-system features, e.g.\ higher-rank
polymorphism to track memory regions, can in practice produce elaborate and
confusing type-error messages.
\eric{I'm sure there's a better example here than higher-rank polymorphism.}
Idris~\cite{christiansen2014reflect} allows the
programmer to supply error handlers, which are given a data structure

representing the compiler's error message and can rewrite it to insert
domain-specific knowledge. Unfortunately, GHC Haskell does not currently support
such a feature.

%% Thus, the EDSL author can provide custom type-error
%% messages using language the application programmers will understand.

The next best thing is to at least give the user accurate error location
information. Dynamically reporting errors in an EDSL is also made difficult by the lack of
source locations. Consider the Ivory expression \hbox{\cd{x / y},} which induces a
runtime assertion \cd{y != 0}. If the assertion fails, we would like to include
a \emph{Haskell} source location in the error message to direct the programmer to the source of
the error. But how are we to obtain the source location in the first place?
Ivory programs are comprised of Haskell expressions and Haskell is
pure--functions cannot depend on their call-site--so we must look outside the
language proper to obtain source locations. A common approach for writing
location-aware functions in Haskell, epitomized by the \cd{file-location}
package~\cite{file-location}, is to use a Template Haskell splice to query the
compiler for the current source location and insert it into the AST. This
approach did not appeal to us as Template Haskell incurs considerable syntactic
overhead, and we already had 10+ KLoC of Ivory which would all need to be
rewritten.

Instead we opted to write a compiler plugin that rewrites GHC's intermediate
representation to add the source locations. The implementation was
straightforward, we first extended the \cd{Ivory} statement type with a new
\cd{Location} constructor that just contains a location in the Haskell source,
essentially a special type of comment. The plugin then extracts the source
locations from GHC and wraps all actions in the \cd{Ivory} monad with a
\cd{withLocation} function that emits a \cd{Location} statement before executing
the wrapped action. While this effectively limits the location granularity to
lines rather than columns, our approach required only modest changes to the
Ivory compiler and--importantly--no changes to Ivory code. As the plugin
itself requires little knowledge of Ivory, we have abstracted it out into a
separate package~\cite{ghc-srcspan-plugin} that can be reused by other projects.
Furthermore, we have since submitted a lightweight extension to GHC that allows
functions to request their call-site by taking a special implicit
parameter~\cite{lewis2000implicit} as an argument, which should become available
in the 7.12 release.

\lee{Lee: describe line number insertions (overcoming a bad thing), module
  system (hack), the use of TH to make a concrete syntax, macros (e.g., standard
  lib---compare to Rust), other uses of macros, small compiler.}


\section{Related Work}
\label{sec:related-work}

The general idea of safe C languages is not new; our main contribution is
embedding a type system into GHC as well as our support of verification
tools.

Pioneering work in the area is the Cyclone language and
compiler~\cite{cyclone}. Cyclone is a dialect of C. Cyclone is less restrictive than
Ivory, relying on both static analysis and runtime checks to enforce memory
safety. Cyclone provides regions for dynamic memory allocation; garbage
collection is optional. Cyclone programs are typically slightly larger than
their C equivalents and mostly syntactically the same. In contrast to Ivory,
Cyclone does not provide macro-programming facilities (beyond the C
preprocessor), nor does it interface to verification and testing
tools. Unfortunately, Cyclone is not actively maintained.

Bit-data and memory areas in Ivory borrow heavily from Diatchki~et~al.'s
previous work~\cite{high-level, memareas}. Indeed, one can consider the present
work as demonstrating the feasibility of embedding this language into Haskell
and GHC types. BitC is another language deprecated research language that
explored a similar design space~\cite{bitc}.

Spark/Ada is a mature language for high-assurance embedded
programming, with a contract language and verification tools to
prove invariants~\cite{spark}. To support verification, the language is
very restrictive; in particular, there are no references in the language.

Rust is an actively-developed safe C language, originating from
Mozilla~\cite{rust}. Rust has a powerful type system to enforce safely using
heap-based data structures. An affine type system prevents pointer aliasing
errors. Rust provides reference counting garbage collection as a library (so
other garbage collection strategies can be used instead). Rust has hygenic
macros. The language has a property-based testing framework, but no mature
support for verification, or even static checks for undefined behavior (e.g.,
division by a constant zero expression).

EDSLs for safe C programming also exist. Examples include Atom, a language for
lock-free embedded programs~\cite{atom}; Copilot, a stream-oriented synchronous
language~\cite{copilot}; SBV, a Haskell-based SMT symbolic simulator with a C code
generator~\cite{sbv}; and Feldspar, a language specialized for high-level and
efficient specifications of digital signal processing~\cite{feldspar1}. Compared to these
languages, Ivory is more focused on the kinds of C in low-level code, such as
device drivers, with bit-data and memory area manipulation.


\section{Conclusion}
\label{sec:conlusion}

\lee{Ivory, yay.}



\lee{summarize the TCs over the AST?}

%% \appendix
%% \section{Appendix Title}



\acks

This work is supported by DARPA under contract no. FA8750-12-9-0169.  Opinions
expressed herein are our own.

% We recommend abbrvnat bibliography style.

\bibliographystyle{abbrvnat}
\bibliography{paper}

% The bibliography should be embedded for final submission.

%% \begin{thebibliography}{}
%% \softraggedright

%% \bibitem[Smith et~al.(2009)Smith, Jones]{smith02}
%% P. Q. Smith, and X. Y. Jones. ...reference text...

%% \end{thebibliography}


\end{document}

%                       Revision History
%                       -------- -------
%  Date         Person  Ver.    Change
%  ----         ------  ----    ------

%  2013.06.29   TU      0.1--4  comments on permission/copyright notices

