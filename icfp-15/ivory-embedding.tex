\section{Ivory Embedding}
\label{sec:ivory-embedding}

\lee{This section will be the heart of the paper, focusing on how we have
  embedded Ivory in Haskell's type system. I expect this to be 3-5 pages and
  probably the most technical. We want to be detailed, but describe things
  general enough that ML, Agda, etc. programmers can understand.
}

\subsection{The Ivory Monad}

\lee{Lee: describe the Ivory monad (not too much to say), then describe
  effects. Make sure to make clear these are monad effects like are popular in
  the literature. Mention inspiration from
  \url{http://www.doc.ic.ac.uk/~wlj05/files/Deconstraining.pdf}, but didn't
  follow implementation.}

\lee{Also talk about the use of typeclasses \cd{IvoryVar}, \cd{IvoryExpr},
  etc. and how this allows the user to define custom data, safely.}

\subsection{Memory Areas}
\lee{Trevor: describe the kind \cd{Area} and their use.}

\paragraph{Stored values}
\lee{Trevor: the warm up for memory areas.}

\paragraph{Structs}
\lee{Trevor: introduce structs and labels. Describe the use of TH here.}

\paragraph{Arrays}
\lee{Trevor: Introduce the use of type-level nats here, and how they are used to
  enforce read/write safety for arrays. I'd discuss \cd{arrayMap} and \cd{Ix}
  here, too. (Maybe discuss possible extensions with decidable type-level
  arithmetic). Maybe mention \cd{CArray}s, too.}

\subsection{References and Allocation}
\lee{Trevor: talk about references, pointers, and stack allocation. }

\subsection{Procedures}
\lee{Trevor: describe the use of type-level lists to define procedures. Describe
the use of a type-class to ensure procedure calls in Ivory are type-correct.}

\subsection{Bit-Data}
\lee{James: talk about using TH to define bitdata.}

\subsection{Module System}
\lee{Lee: describe the module system, warts and all.}
